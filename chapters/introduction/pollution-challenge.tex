\section[La sfida dell'inquinamento]{La sfida dell'inquinamento\footcite[I dati
forniti in questo paragrafo sono prelevati da][]{legambiente2019}}

L'inquinamento dell'aria è una delle principali sfide ambientali dell'epoca
attuale perché le sostanze inquinanti presenti nell'aria sono dannose per
l'ecosistema e per la salute umana.

Questo problema è particolarmente avvertito anche in Italia: nel 2018, ad
esempio, i limiti giornalieri previsti dalla legge per le polveri sottili
(\(\chem{PM}\)) e l'ozono (\(\chem{O_3}\)) sono stati oltrepassati in \emph{ben
55 capoluoghi di provincia}. Brescia, nello stesso anno, ha registrato il record
di numero di giornate ``fuorilegge'' con \(150\) giorni oltre i limiti legali e
i decessi prematuri dovuti all'inquinamento sul territorio nazionale nel 2015
sono stati oltre \(60600\), uno dei dati peggiori in Europa.

Tra gli inquinanti più pericolosi troviamo il \standout{particolato atmosferico
sottile} (\(\chem{PM2.5}\)) e gli \standout{ossidi di azoto} (\(\chem{NO}\) e
\(\chem{NO_2}\)). Il primo, prodotto prevalentemente da industrie, riscaldamento
e traffico veicolare, è costituto da particelle solide e liquide che influiscono
sulla propagazione e sull'assorbimento delle radiazioni solari mettendo a
rischio l'ecosistema; inoltre, se inalate possono causare gravi danni al sistema
respiratorio, circolatorio e immunitario. Il particolato atmosferico può essere
di origine primaria, se rilasciato nell'atmosfera direttamente dalla sorgente, o
secondaria, se si produce a causa di trasformazioni chimico-fisiche a partire da
altri inquinanti. Gli ossidi di azoto, prodotti prevalentemente dal traffico
urbano (sopratutto dei motori diesel), sono invece molto pericolosi per i
soggetti più deboli come i bambini, per i malati di asma e in generale per chi
soffre di malattie respiratorie.

Un altro inquinante molto pericoloso, di origine secondaria e spesso
sottovalutato, è l'\standout{ozono troposferico} (\(\chem{O_3}\)) che, oltre a
causare danni al sistema respiratorio (causa ogni anno in Europa di oltre
\(17700\) morti), ha significativi effetti sull'ecosistema e sull'agricoltura.
L'\(\chem{O_3}\) non è prodotto direttamente da nessuna sorgente ma si forma,
prevalentemente in estate, attraverso trasformazioni chimico-fisiche di altre
sostanze tra cui, sopratutto, gli ossidi di azoto.

Per gli effetti dannosi sopra elencati, è fondamentale che \emph{la riduzione
degli agenti inquinanti nell'aria sia un obiettivo prioritario di tutte le
amministrazioni territoriali} con piani volti a ridurre il traffico veicolare e
le altre fonti di inquinamento. Sfortunatamente però, in Italia, ancora oggi si
registra \emph{uno dei tassi più alti di motorizzazione d'Europa} (circa 65 auto
ogni 100 abitanti) e i mezzi di trasporto collettivo avrebbero bisogno di
investimenti volti a migliorare la copertura del territorio e la qualità
del servizio al fine di indurre più utenti ad utilizzare tali mezzi in
alternativa ai veicoli privati.

Nella Tabella \vref{tab:illegaldaysprovinces} sono illustrati i giorni totali in
cui i limiti di legge previsti per l'emissione di ozono e polveri sottili sono
stati superati nei capoluoghi di provincia italiani nel 2018.
\begin{table}
	\caption{giorni totali di superamento dei limiti previsti per le polveri
	sottili (\(\chem{PM10}\)) o per l'ozono nei capoluoghi di provincia
	italiani nell'anno solare 2018.}\label{tab:illegaldaysprovinces}
	\bigskip

	\begin{minipage}{0.35\textwidth}
		\begin{tabular}{r|l}
			\hline
			Brescia & \badvalue{150} \\
			\hline
			Lodi & \badvalue{149} \\
			\hline
			Monza & \badvalue{140} \\
			\hline
			Venezia & \badvalue{139} \\
			\hline
			Alessandria & \badvalue{136} \\
			\hline
			Milano & \badvalue{135} \\
			\hline
			Torino & \badvalue{134} \\
			\hline
			Padova & \badvalue{130} \\
			\hline
			Bergamo & \badvalue{127} \\
			\hline
			Cremona & \badvalue{127} \\
			\hline
			Rovigo & \badvalue{121} \\
			\hline
			Modena & \badvalue{117} \\
			\hline
			Treviso & \badvalue{116} \\
			\hline
			Frosinone & \badvalue{116} \\
			\hline
			Pavia & \badvalue{115} \\
			\hline
			Verona & \badvalue{114} \\
			\hline
			Asti & \badvalue{113} \\
			\hline
			Parma & \badvalue{112} \\
			\hline
			Reggio Emilia & \badvalue{111} \\
			\hline
		\end{tabular}
	\end{minipage}
	\hfill
	\begin{minipage}{0.29\textwidth}
		\begin{tabular}{r|l}
			\hline
			Genova & 103 \\
			\hline
			Avellino & \badvalue{89} \\
			\hline
			Lecco & 88 \\
			\hline
			Terni & \badvalue{86} \\
			\hline
			Rimini & \badvalue{82} \\
			\hline
			Vicenza & \badvalue{82} \\
			\hline
			Piacenza & 80 \\
			\hline
			Varese & 78 \\
			\hline
			Roma & 72 \\
			\hline
			Napoli & \badvalue{72} \\
			\hline
			Mantova & 65 \\
			\hline
			Lucca & 61 \\
			\hline
			Forlì & 48 \\
			\hline
			Firenze & 45 \\
			\hline
			Grosseto & 44 \\
			\hline
			Pordenone & 44 \\
			\hline
			Como & 43 \\
			\hline
			Biella & 42 \\
			\hline
			Ravenna & 42 \\
			\hline
		\end{tabular}
	\end{minipage}
	\hfill
	\begin{minipage}{0.27\textwidth}
		\begin{tabular}{r|l}
			\hline
			Vercelli & 41 \\
			\hline
			Ferrara & 41 \\
			\hline
			Bologna & 39 \\
			\hline
			Trento & 38 \\
			\hline
			Udine & 37 \\
			\hline
			Sondrio & 35 \\
			\hline
			Pisa & 32 \\
			\hline
			Trieste & 32 \\
			\hline
			Macerata & 31 \\
			\hline
			Rieti & 31 \\
			\hline
			Savona & 28 \\
			\hline
			Aosta & 27 \\
			\hline
			Benevento & 27 \\
			\hline
			Pistoia & 27 \\
			\hline
			Agrigento & 26 \\
			\hline
			Bolzano & 26 \\
			\hline
			Enna & 26 \\
			\hline
			 & \\
			\hline
			 & \\
			\hline
		\end{tabular}
	\end{minipage}
	\par
	\bigskip
	\footnotesize
	\textit{Fonte: elaborazione Legambiente su dati ARPA o Regioni.}
	\par
	\textit{\standout{NB:} \badvalue{in rosso} i giorni totali di
	superamento delle città in cui si è registrato nel 2018 sia il
	superamento dei limiti del \(\chem{PM10}\) che dell'ozono. In nero i
	giorni di superamento del limite previsto per l'ozono (25 giorni
	all'anno); per la città di Ferrara si riportano i giorni di superamento
	previsti per le polveri sottili (35 giorni all'anno).}
\end{table}
