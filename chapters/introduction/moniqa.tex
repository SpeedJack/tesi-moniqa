\section{MonIQA}

Per affrontare i problemi presentati nel capitolo precedente, è importante che
le amministrazioni si dotino di strumenti di analisi della qualità dell'aria al
fine di valutare quali sono le aree, i periodi e gli inquinanti di maggior
interesse e di verificare la buona riuscita delle scelte e degli investimenti
adottati.

A tal scopo nasce \MonIQA\@. Questo servizio, sviluppato dall'Università di Pisa
con il lavoro del Prof.\ Giuseppe Anastasi e degli Ing.\ Francesca Righetti e
Carlo Vallati, offre la possibilità di visualizzare i dati sulla qualità
dell'aria provenienti da stazioni di rilevamento dislocate sul territorio
nazionale\footnote{I dati sono pubblicati sui siti web dalle Agenzie Regionali
per la Protezione dell'Ambiente (ARPA) e quindi raccolti da \MonIQA\@.}.

Per facilitare il monitoraggio della qualità dell'aria, \MonIQA{} utilizza un
indice che permette di rappresentare in modo sintetico la qualità dell'aria
valutando contemporaneamente i dati disponibili su vari inquinanti atmosferici.
Tale indice, chiamato \standout{Indice di Qualità dell'Aria} (IQA), viene
calcolato dividendo la \emph{misura} relativa all'inquinante maggiormente
presente (relativamente ai suoi limiti di legge) nell'area di interesse per il
suo \emph{limite di riferimento} stabilito dal \mbox{D.\,Lgs.}~155/2010. Ad
esempio: un IQA del \(50\%\) indica che il peggior inquinante ha concentrazioni
pari alla metà del suo limite legale; un IQA del \(150\%\) indica che il peggior
inquinante ha concentrazioni pari a una volta e mezzo del limite.

\MonIQA{} quindi associa delle classi di giudizio e relativi cromatismi in base
al valore dell'IQA:
\begin{itemize}
	\item [\textbf{Buona} \colorbullet{goodiqa}] indica che l'inquinante
		peggiore ha concentrazioni \emph{inferiori alla metà del limite}
		di legge (\(IQA \leq 50\%\));
	\item [\textbf{Discreta} \colorbullet{reasonableiqa}] l'inquinante
		peggiore ha concentrazioni \emph{inferiori al limite} consentito
		(\(50\% < IQA \leq 100\%\));
	\item [\textbf{Mediocre} \colorbullet{mediocreiqa}] l'inquinante
		peggiore ha concentrazioni \emph{fino a una volta e mezzo il
		valore limite} (\(100\% < IQA \leq 150\%\));
	\item [\textbf{Scadente} \colorbullet{pooriqa}] l'inquinante peggiore ha
		concentrazioni \emph{fino a due volte il limite} legale (\(150\%
		< IQA \leq 200\%\));
	\item [\textbf{Pessima} \colorbullet{badiqa}] l'inquinante peggiore ha
		concentrazioni \emph{superiori al doppio del limite} consentito.
\end{itemize}
Queste 5 classi di giudizio permettono di comprendere facilmente e
immediatamente lo stato della qualità dell'aria di un'area o città del
territorio nazionale.

Le amministrazioni territoriali possono quindi utilizzare questo servizio per
guidare le proprie scelte e investimenti nella lotta all'inquinamento
atmosferico e, a posteriori, per verificare i risultati ottenuti. I cittadini,
grazie alla semplicità di comprensione dell'IQA così rappresentato, possono
visualizzare facilmente lo stato di qualità dell'aria della zona in cui si
trovano, invece che andare ad analizzare un report ogni volta diverso per ogni
sito dell'ARPA.

Informare i cittadini e fornire alle amministrazione gli strumenti di analisi
adeguati è \emph{il primo passo fondamentale per affrontare il problema
dell'inquinamento dell'aria}: senza un approccio basato su dati reali e senza il
supporto della popolazione sarebbe difficile individuare e attuare efficaci
piani di sviluppo e di investimento volti alla riduzione delle sostanze nocive
nell'atmosfera.
