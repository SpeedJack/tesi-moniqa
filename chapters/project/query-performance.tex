\section{Performance della query}

La query presentata nel paragrafo precedente ha però il difetto di avere
performance scarse. Ad esempio, eseguendo la query per la provincia di Roma
vediamo che impiega 8 minuti e 5 secondi per fornire il risultato. Tempo
ovviamente non accettabile per un'applicazione web.

Per ridurre il tempo richiesto per l'esecuzione della query si è quindi deciso
di aggiungere una nuova tabella al database che contiene già i dati aggregati
necessari e si aggiorna, attraverso \textit{trigger} MySQL, ogni volta che viene
aggiunta, modificata, eliminata una misurazione.

La nuova tabella, chiamata \code{historical\_data}, è definita come mostrato nel
Listato \ref{lst:historicaldatatable}.

\lstinputlisting[language=SQL, label={lst:historicaldatatable}, caption={Schema
della tabella \code{historical\_data} e dell'istruzione \code{INSERT} utilizzata
per riempire la tabella con tutte le misurazioni già presenti nel
database.}]{historical-data-table.sql}

I campi \code{year}, \code{quarter} e \code{fk\_sensortype} contengono,
rispettivamente, l'anno, il quarto di anno e la chiave esterna al tipo di
sensore. Il campo \code{fk\_city} contiene la chiave esterna alla provincia. I
campi \code{measure\_sum} e \code{measure\_count} contengono invece la somma di
tutte le misurazioni (per quella provincia, quarto di anno e tipo di sensore) e
il numero totale di misurazioni disponibili.  La media può essere calcolata come
\(measure\_sum / measure\_count\). In questo modo abbiamo realizzato una tabella
che fa da ``cache'' per i dati aggregati che vogliamo ottenere e possiamo
utilizzare la query mostrata nel Listato \ref{lst:fastquery} per prelevare i
dati.

\lstinputlisting[language=SQL, label={lst:fastquery}, caption={Query ottimizzata
per il prelievo dei dati dal database.}]{fast-query.sql}

Le performance sono molto migliorate. Infatti la nuova query, sempre per la
provincia di Roma, ritorna il risultato in meno di un centesimo di secondo.

Tre trigger MySQL si occupano di mantenere aggiornata tale tabella ad ogni
operazione di \code{INSERT}, \code{UPDATE}, \code{DELETE} sulla tabella
\code{measurements}. L'aggiornamento non richiede alcuna operazione
\textit{resource-intensive} in quanto, ad esempio, per l'inserimento di una nuova
misurazione, è sufficiente individuare il record da aggiornare (in base al
periodo, alla provincia e al tipo di sensore della nuova misurazione), quindi
sommare a \code{measure\_sum} il valore della nuova misurazione e incrementare
di uno il contatore \code{measure\_count}.
