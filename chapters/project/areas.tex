\section{Aree geografiche}\label{sec:areas}

La prima questione che si pone per la realizzazione del progetto, riguarda come
definire le aree geografiche che l'utente può selezionare come area da
analizzare. Trovare una soluzione a questo problema si rende necessario in
quanto, talvolta, nel database di \MonIQA{}, per alcune stazioni di
monitoraggio, sono presenti periodi di tempo in cui non sono disponibili
misurazioni.  Pertanto, si può limitare il problema aggregando insieme più
stazioni di monitoraggio. I dati aggregati così raccolti avranno minor
possibilità di contenere periodi privi di dati (risulta infatti improbabile che
mettendo insieme le informazioni provenienti da più stazioni ci siano intervalli
di tempo dove nessuna stazione ha raccolto dati). Le cause di questi vuoti di
dati possono essere varie: alcune stazioni di monitoraggio possono andare
\textit{offline} per alcuni periodi, oppure gli script che si occupano di
prelevare le misurazioni e inserirle nel database di \MonIQA{} possono smettere
di funzionare, e altro ancora.

Altro motivo per cui è importante definire aree geografiche è che l'obiettivo
del progetto è \emph{consentire l'analisi storica della qualità dell'aria di una
zona più o meno vasta}, e non solo di un singolo punto \idest{stazione di
monitoraggio}.

Pertanto, le aree geografiche da definire dovranno rispettare i seguenti
requisiti.
\begin{itemize}
	\item Devono essere \emph{sufficientemente vaste} da evitare troppi
		periodi privi di misurazioni nei dati aggregati.
	\item \emph{Non devono essere troppo vaste} per evitare di aggregare
		dati da stazioni di monitoraggio troppo distanti che,
		altrimenti, fornirebbero misurazioni di zone differenti del
		territorio nazionale, perdendo la validità di tali informazioni.
	\item Devono complessivamente \emph{coprire l'intero territorio
		nazionale}.
	\item Devono avere un \emph{nome chiaro che identifichi la zona} che si
		vuole analizzare. L'utente, infatti, dovrà poter selezionare
		l'area di suo interesse a partire dal nome.
\end{itemize}

Per rispondere a tali requisiti si è deciso di utilizzare \emph{le province e le
città metropolitane} (che chiameremo indifferentemente ``province'' nel
seguito).  Infatti, questi enti territoriali, non sono né eccessivamente vasti
né troppo piccoli e hanno tutti il nome di una città capoluogo che identifica
chiaramente l'area geografica. Inoltre, dato che il servizio ha anche
l'obiettivo di essere utile alle amministrazioni territoriali per valutare piani
e investimenti da attuare per ridurre l'inquinamento, una suddivisione basata
sui territori di competenza delle amministrazioni provinciali risulta
ragionevole.

Nel database di \MonIQA{} per ogni stazione di monitoraggio è presente un campo
\idest{colonna} ``città'' che relaziona la stazione con la città in cui si trova
(una descrizione dettagliata del database è fornita nel paragrafo successivo).
Nel database, gran parte dei record contengono il valore \code{NULL} in questo
campo, in quanto non tutte le stazioni si possono chiaramente assegnare a una
città (alcune stazioni si trovano in zone periferiche). Pertanto, al fine di
rimuovere questi valori \code{NULL} così da avere ogni stazione con una città
associata, si è deciso di utilizzare questo stesso campo della tabella delle
stazioni per associarle alle province: quindi ogni stazione non avrà più una
città di appartenenza, ma una provincia.

Nel database sono presenti quasi \(1000\) stazioni di monitoraggio, sarebbe
quindi scomodo assegnarle manualmente una per una alla relativa provincia.
Pertanto si è deciso di utilizzare uno script scritto in linguaggio Python
(\code{fill-cities.py}) che:
\begin{enumerate}
	\item preleva dal database la latitudine e la longitudine di ogni
		stazione di monitoraggio;
	\item utilizza un servizio esterno\footnote{Si è deciso di utilizzare il
		servizio \textsc{MapIt} fornito dall'Associazione Openpolis
		all'indirizzo \url{http://mapit.openpolis.it}.  Tale piattaforma
		fornisce una REST API che permette di ottenere, in formato JSON,
		informazioni sulle aree amministrative che circondano un punto
		geografico fornito tramite parametri di una richiesta HTTP
		GET\@. \textsc{MapIt} utilizza i dati sui confini delle unità
		amministrative rilasciati pubblicamente ogni anno dall'ISTAT.}
		per determinare la provincia in cui si trova la stazione di
		monitoraggio;
	\item produce uno script SQL che contiene le istruzioni \code{INSERT}
		che aggiungono le province nell'apposita tabella del database e
		le istruzioni \code{UPDATE} che aggiornano il campo ``città''
		delle stazioni in modo da associarlo alla provincia a cui
		appartengono.
\end{enumerate}

Si è deciso di generare uno script SQL piuttosto che modificare direttamente il
database, così da consentire all'utilizzatore dello script Python di revisionare
il risultato prima di eseguirlo sul server.

Lo script Python stampa le istruzioni SQL su stream \textit{standard output},
mentre sullo stream \textit{standard error} fornisce informazioni su quanto lo
script sta facendo (la stazione che sta processando e la provincia individuata)
e su eventuali errori.  In tal modo, è possibile utilizzare le funzioni di
redirezione dell'output fornite dal sistema operativo per produrre lo script SQL
su file senza rinunciare a controllare l'andamento dell'esecuzione.
